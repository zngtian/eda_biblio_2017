% Created 2017-08-22 Tue 15:29
% -*- latex-run-command: xelatex -*-
\documentclass[a4paper,11pt]{article}
\usepackage{graphicx}
\usepackage{grffile}
\usepackage{longtable}
\usepackage{wrapfig}
\usepackage{rotating}
\usepackage[normalem]{ulem}
\usepackage{amsmath}
\usepackage{textcomp}
\usepackage{amssymb}
\usepackage{capt-of}
\usepackage{hyperref}
\usepackage[margin=1in]{geometry}
\usepackage{setspace}
\onehalfspacing
\usepackage{parskip}
\usepackage{tabularx}
\usepackage{color}
\usepackage{caption}
\usepackage{subcaption}
\usepackage[round]{natbib}
\hypersetup{colorlinks,citecolor=black,filecolor=black,linkcolor=black,urlcolor=black}
\setcounter{secnumdepth}{2}
\author{Zheng Tian, Jing Chen}
\date{\today}
\title{Annotated Bibliography (Updated)}
\hypersetup{
 pdfauthor={Zheng Tian, Jing Chen},
 pdftitle={Annotated Bibliography (Updated)},
 pdfkeywords={},
 pdfsubject={},
 pdfcreator={Emacs 25.1.1 (Org mode 9.0.9)}, 
 pdflang={English}}
\begin{document}

\maketitle


\section{\textcolor{red}{TODO} Introduction}
 \label{introduction}
 
\section{\textcolor{red}{TODO} Literature by Category}
\label{lit_category}

\subsection{General Theory}

\begin{enumerate}
\item Breschi, S., \& Malerba, F. (1997). Sectoral Innovation Systems: Technological Regimes, Schumpeterian Dynamics and Spatial Boundaries. In C. Edquist (Ed.), Systems of Innovation: Technologies, Institutions and Organizations (pp. 130–156). Pinter Publishers.
\item Burrus, D., \& Gittines, R. (1993). Technotrends: How to Use Technology to Go Beyond Your Competition (1st edition). New York: Harpercollins.
\item Bush, V. (1960). Science, the endless frontier: a report to the President on a program for postwar scientific research.
\item Carlsson, B., Jacobsson, S., Holmén, M., \& Rickne, A. (2002). Innovation systems: analytical and methodological issues. Research Policy, 31(2), 233–245. https://doi.org/10.1016/S0048-7333(01)00138-X
\item Edquist, C. (Ed.). (2012). Systems of Innovation: Technologies, Institutions and Organizations (1 edition). London: Routledge.
\item Freeman, C. (1987). Technology Policy and Economic Performance: Lessons from Japan. London ; New York: Pinter Pub Ltd.
\item Freeman, C., \& Soete, L. (2009). Developing science, technology and innovation indicators: What we can learn from the past. Research Policy, 38(4), 583–589. https://doi.org/10.1016/j.respol.2009.01.018
\item Kline, S. J., \& Rosenberg, N. (1986). An Overview of Innovation. In R. Landau \& N. Rosenberg (Eds.), The positive sum strategy: Harnessing technology for economic growth (pp. 275–305). Washington, D.C.
\item Lundvall, B.-Å. (Ed.). (2010). National Systems of Innovation: Toward a Theory of Innovation and Interactive Learning (Revised ed. edition). London: Anthem Press.
\item Malecki, E. J. (1997). Technology and Economic Development: The Dynamics of Local, Regional and National Competitiveness (2 edition). Essex, England: Longman Pub Group.
\item Malerba, F. (Ed.). (2004). Sectoral Systems of Innovation: Concepts, Issues and Analyses of Six Major Sectors in Europe. New York, N.Y: Cambridge University Press.
\item National Science Foundation (U.S.). (1957). Basic Research: A National Resource. Washington, D.C.: National Science Foundation. Retrieved from https//catalog.hathitrust.org/Record/006685493
\item Nelson, R. R. (1959). The Simple Economics of Basic Scientific Research. Journal of Political Economy, 67, 297–297.
\item Nelson, R. R. (Ed.). (1993). National Innovation Systems: A Comparative Analysis (1 edition). New York: Oxford University Press.
\item Oinas, P., \& Malecki, E. J. (2002). The Evolution of Technologies in Time and Space: From National and Regional to Spatial Innovation Systems. International Regional Science Review, 25(1), 102–131. https://doi.org/10.1177/016001702762039402
\item Rosenberg, N. (1983). Inside the Black Box: Technology and Economics. Cambridge Cambridgeshire ; New York: Cambridge University Press.
\item Soete, L. (2007). From Industrial to Innovation Policy. Journal of Industry, Competition and Trade, 7(3–4), 273. https://doi.org/10.1007/s10842-007-0019-5
\item Stokes, D. E. (1997). Pasteur’s Quadrant: Basic Science and Technological Innovation. Washington, D.C: Brookings Institution Press.
\end{enumerate}

\subsection{Regional Innovation}

\begin{enumerate}
\item Alcaide-Marzal, J., \& Tortajada-Esparza, E. (2007). Innovation assessment in traditional industries. A proposal of aesthetic innovation indicators. Scientometrics, 72(1), 33–57. https://doi.org/10.1007/s11192-007-1708-x
\item Asheim, B. T., \& Isaksen, A. (2002). Regional Innovation Systems: The Integration of Local ‘Sticky’ and Global ‘Ubiquitous’ Knowledge. The Journal of Technology Transfer, 27(1), 77–86. https://doi.org/10.1023/A:1013100704794
\item Autio, E. (1998). Evaluation of RTD in regional systems of innovation. European Planning Studies, 6(2), 131–140. https://doi.org/10.1080/09654319808720451
\item Carlsson, B., Jacobsson, S., Holmén, M., \& Rickne, A. (2002). Innovation systems: analytical and methodological issues. Research Policy, 31(2), 233–245. https://doi.org/10.1016/S0048-7333(01)00138-X
\item Christopherson, S., \& Clark, J. (2007). Power in Firm Networks: What it Means for Regional Innovation Systems. Regional Studies, 41(9), 1223–1236. https://doi.org/10.1080/00343400701543330
\item Colapinto, C. (2007). A way to foster innovation: a venture capital district from Silicon Valley and route 128 to Waterloo Region. International Review of Economics, 54(3), 319–343. https://doi.org/10.1007/s12232-007-0018-1
\item Cooke, P. (2001). Regional Innovation Systems, Clusters, and the Knowledge Economy. Industrial and Corporate Change, 10(4), 945–974. https://doi.org/10.1093/icc/10.4.945
\item Cooke, P., Gomez Uranga, M., \& Etxebarria, G. (1997). Regional innovation systems: Institutional and organisational dimensions. Research Policy, 26(4), 475–491. https://doi.org/10.1016/S0048-7333(97)00025-5
\item Cooke, P. N., Heidenreich, M., \& Braczyk, H.-J. (2004). Regional Innovation Systems: The Role of Governance in a Globalized World. Psychology Press.
\item Evangelista, R., Iammarino, S., Mastrostefano, V., \& Silvani, A. (2002). Looking for Regional Systems of Innovation: Evidence from the Italian Innovation Survey. Regional Studies, 36(2), 173–186. https://doi.org/10.1080/00343400220121963
\item Freeman, C. (2002). Continental, national and sub-national innovation systems—complementarity and economic growth. Research Policy, 31(2), 191–211. https://doi.org/10.1016/S0048-7333(01)00136-6
\item Krauss, G., \& Wolf, H.-G. (2002). Technological Strengths in Mature Sectors--An Impediment or an Asset for Regional Economic Restructuring? The Case of Multimedia and Biotechnology in Baden-Wurttemberg. The Journal of Technology Transfer, 27(1), 39–50. https://doi.org/10.1023/A:1013144519815
\item Liu, S., \& Chen, C. (2003). Regional innovation system: Theoretical approach and empirical study of China. Chinese Geographical Science, 13(3), 193–198. https://doi.org/10.1007/s11769-003-0016-5
\item Scott, A. J. (2006). Entrepreneurship, Innovation and Industrial Development: Geography and the Creative Field Revisited. Small Business Economics, 26(1), 1–24. https://doi.org/10.1007/s11187-004-6493-9
\item Simmie, J. (2003). Innovation and Urban Regions as National and International Nodes for the Transfer and Sharing of Knowledge. Regional Studies, 37(6–7), 607–620. https://doi.org/10.1080/0034340032000108714
\end{enumerate}

\subsection{Methodology}

\begin{enumerate}
\item Acs, Z. J., Anselin, L., \& Varga, A. (2002). Patents and innovation counts as measures of regional production of new knowledge. Research Policy, 31(7), 1069–1085.
\item Acs, Z. J., \& Audretsch, D. B. (1993). Analysing Innovation Output Indicators: The US Experience. In A. Kleinknecht \& D. Bain (Eds.), New concepts in innovation output measurement (pp. 10–41). Palgrave Macmillan UK. https://doi.org/10.1007/978-1-349-22892-8-2
\item Arundel, A. (2007). Innovation Survey Indicators: What Impact on Innovation Policy? In D. Organisation for Economic Co-operation and (Ed.), Science, Technology and Innovation Indicators in a Changing World: Responding to Policy Needs.
\item Edquist, C. (1997). Systems of innovation: technologies, institutions, and organizations. Routledge. https://doi.org/10.4324/9780203357620
\item Evangelista, R., \& et al. (2002). Looking for Regional Systems of Innovation: Evidence from the Italian Innovation Survey. Regional Studies, 36(2), 173–186.
\item Gertler, M. S., Wolfe, D. A., \& Garkut, D. (1998). The Dynamics of Regional Innovation in Ontario. In Local and Regional Systems of Innovation (pp. 211–238). Boston, MA: Springer, Boston, MA. https://doi.org/10.1007/978-1-4615-5551-3-11
\item Griliches, Z. (1990). Patent Statistics as Economic Indicators - A Survey. Journal of Economic Literature, 28(4), 1661–1707.
\item Grupp, H., \& Mogee, M. E. (2004). Indicators for national science and technology policy: How robust are composite indicators? Research Policy, 33(9), 1373–1384. https://doi.org/10.1016/j.respol.2004.09.007
\item Hall, J. L. (2008). Adding Meaning to Measurement. Economic Development Quarterly, 23(1), 3–12. https://doi.org/10.1177/0891242408326467
\item Hall, J. L. (2016). Developing Historical 50-State Indices of Innovation Capacity and Commercialization Capacity. Economic Development Quarterly, 21(2), 107–123. https://doi.org/10.1177/0891242406298128
\item Kleinknecht, A., \& Van Montfort, K. (2002). The non-trivial choice between innovation indicators. Economics of Innovation, 11(2), 109–121. https://doi.org/10.1080/10438590210899
\item NSF. (1956). Expenditures for R\&D in the United States 1953. Washington, D.C.: National Science Foundation.
\item OECD. (1963). Proposed Standard Practice for Surveys of Research and Development. Paris: Directorate for Scientific Affairs. OECD.
\item OECD. (1992). Oslo Manual: Proposed Guidelines for Collecting and Interpreting Technological Innovation Data.
\item OECD. (1997). National Innovation Systems. Organization for Economic Cooperation and Development.
\item OECD, \& European Communities Statistical Office. (2005). Oslo Manual: Proposed Guidelines for Collecting and Interpreting Technological Innovation Data. OECD/Eurostat.
\item Porter, M., \& Stern, S. (1999). The New Challenge to America’s Prosperity: Findings from the Innovation Index. Washington, D.C.: Council on Competitiveness.
\item Sajeva, M., \& Gatelli, D. (2005). Methodology Report on European Innovation Scoreboard 2005. European Commission, Enterprise Directorate-General.
\item Simmie, J. (2003). Innovation and urban regions as national and international nodes for the transfer and sharing of knowledge. Regional Studies, 37(6–7), 607–620. https://doi.org/10.1080/0034340032000108714
\item Smith, K. H. (2005). Measuring innovation. In J. Fagerberg \& D. C. Mowery (Eds.), The Oxford Handbook of Innovation. Oxford University Press. https://doi.org/10.1093/oxfordhb/9780199286805.003.0006
\item Tijssen, R. (2003). Scoreboards of research excellence. Research Evaluation, 12(2), 91–103. https://doi.org/10.3152/147154403781776690
\end{enumerate}

\subsection{Application}

\begin{enumerate}
\item Arundel, A. (2007). Innovation Survey Indicators: What Impact on Innovation Policy? In D. Organisation for Economic Co-operation and (Ed.), Science, Technology and Innovation Indicators in a Changing World: Responding to Policy Needs.
\item Asheim, B. T., \& Isaksen, A. (2002). Regional Innovation Systems: The Integration of Local “Sticky” and Global “Ubiquitous” Knowledge. Journal of Technology Transfer, 27(1), 77–86.
\item Cooke, P., Heidenreich, M., \& Braczyk, H. J. (2004). Regional Innovation Systems: The Role of Governance in a Globalized World. New York: Routledge.
\item Cooke, P., \& Memedovic, O. (2003). Strategies for Regional Innovation Systems: Learning Transfer and Applications. Vienna, Austria: United Nations Industrial Development Organization.
\item Diez, J. R. (2002). Metropolitan innovation systems: A comparison between Barcelona, Stockholm, and Vienna. International Regional Science Review, 25(1), 63–85.
\item Evangelista, R., \& et al. (2002). Looking for Regional Systems of Innovation: Evidence from the Italian Innovation Survey. Regional Studies, 36(2), 173–186.
\item Fischer, M. M., Revilla Diez, J., \& Snickars, F. (2001). Metropolitan innovation systems: Theory and evidence from three metropolitan regions in Europe. In association with Attila Varga. Advances in Spatial Science. Heidelberg and New York: Springer.
\item Grupp, H., \& Mogee, M. E. (2004). Indicators for national science and technology policy: How robust are composite indicators? Research Policy, 33(9), 1373–1384. https://doi.org/10.1016/j.respol.2004.09.007
\item Hall, J. L. (2008). Adding Meaning to Measurement. Economic Development Quarterly, 23(1), 3–12. https://doi.org/10.1177/0891242408326467
\item Hall, J. L. (2016). Developing Historical 50-State Indices of Innovation Capacity and Commercialization Capacity. Economic Development Quarterly, 21(2), 107–123. https://doi.org/10.1177/0891242406298128
\item Holbrook, A., \& Salazar, M. (2004). Regional Innovation Systems Within A Federation: Do national policies affect all regions equally? Innovation: Management, Policy \& Practice, 6(1), 50–64.
\item Isaksen, A. (2001). Building Regional Innovation Systems: Is Endogenous Industrial Development Possible in the Global Economy? Canadian Journal of Regional Science, 24(1), 101–120.
\item Pavitt, K., Robson, M., \& Townsend, J. (1987). The Size Distribution of Innovating Firms in the UK - 1945-1983. Journal of Industrial Economics, 35(3), 297–316.
\item Porter, M., \& Stern, S. (1999). The New Challenge to America’s Prosperity: Findings from the Innovation Index. Washington, D.C.: Council on Competitiveness.
\item Simmie, J. (2003). Innovation and urban regions as national and international nodes for the transfer and sharing of knowledge. Regional Studies, 37(6–7), 607–620. https://doi.org/10.1080/0034340032000108714
\item Soete, L. (2006). Knowledge, policy and innovation. In L. Earl \& F. Gault (Eds.), National Innovation, Indicators and Policy (pp. 198–218). Cheltenham: Edward Elgar.
\item Tijssen, R. (2003). Scoreboards of research excellence. Research Evaluation, 12(2), 91–103. https://doi.org/10.3152/147154403781776690
\end{enumerate}

\section{\textcolor{red}{TODO} Annotated Bibliography}
\label{annotated_bib}

Acs, Z. J., Anselin, L., & Varga, A. (2002). Patents and innovation counts as measures of regional production of new knowledge. Research Policy, 31(7), 1069-1085.
Abstract
The role of geographically mediated knowledge externalities in regional innovation systems has become a major issue in research policy. Although the process of innovation is a crucial aspect of economic growth, the problem of measuring innovation has not yet been completely resolved. A central problem involved in such analysis is the measurement of economically useful new knowledge. In the US information on this has been limited to an innovation count data base. Determining the extent to which the innovation data can be substituted
by other measures is essential for a deeper understanding of the dynamics involved. We provide an exploratory and a regression-based comparison of the innovation count data and data on patent counts at the lowest possible levels of geographical aggregation.
Acs, Z. J., & Audretsch, D. B. (1993). Analysing Innovation Output Indicators: The US Experience. In A. Kleinknecht & D. Bain (Eds.), New concepts in innovation output measurement. (pp. 10-41): New York: St. Martin's Press; London: Macmillan Press.
Abstract
Conventional wisdom about innovation was based on studies using the measure of input of innovation, such as R&D expenditures, and the measure of the intermediate output in the process, such as the number of patented inventions.
Recently, some new learning regarding technological change has emerged based on new data sources for the direct measure of innovative output. The purpose of this article is to summarize what has been learned from these new data sources,
 


providing a direct measure of innovative output for the United States, and how these new measures have led to a new learning about the process of technological change.
Alcaide-Marzal, J., and Tortajada-Esparza, E. (2007). Innovation assessment in traditional industries. A proposal of aesthetic innovation indicators.
Scientometrics, 72(1), 33-57.

Abstract
The authors take what could be considered a radical approach toward the assessment of innovation, proposing to include aesthetic, or non-technological, indicators of innovation. Aesthetic innovation occurs when “novelty is conferred on a product in terms of visual” or “sensory” attributes. The study criticizes the Oslo Manual (OECD 1997) the basis for the European Community Innovation Survey (CIS). The emphasis of the article is “traditional” industries, and the study is based on Spain’s footwear industry. The authors’ arguments may be useful in parts of the US where traditional industries such as agriculture remain strong and where, for example, growth in organic food production includes little or no “technological innovation” but holds significant economic potential.
Arundel, A. (2007). Innovation Survey Indicators: What Impact on Innovation Policy? In
D.	Organisation for Economic Co-operation and (Ed.), Science, Technology and Innovation Indicators in a Changing World: Responding to Policy Needs (pp. 49-64). Paris and Washington, D.C.: Organisation for Economic Co-operation and Development.
Abstract
Being first introduced in 1993, the Community Innovation Survey (CIS) is considered one of the most comprehensive major sources of new innovation data at the time. However, in practice, European policy relies more on long-established data for R&D than the CIS. This article is to examine why R&D indicators still dominate innovation policy making in Europe and to make several suggestion for improving the usefulness of the CIS. This requires returning to some of the
 


original goals of the CIS and using the CIS to construct new indicators that better meet the needs of the policy community. Several examples of new indicators are provided, including an output measure with better international comparability, an indicator for knowledge diffusion, and a set of indicators for firms’ innovative capabilities.
Asheim, B. T., & Isaksen, A. (2002). Regional Innovation Systems: The Integration of Local 'Sticky' and Global 'Ubiquitous' Knowledge. Journal of Technology Transfer, 27(1), 77-86.
Abstract
The paper examines how firms in three regional clusters in Norway dominated by shipbuilding, mechanical engineering and electronics industry, respectively exploit both place-specific local resources as well as external, world-class knowledge to strengthen their competitiveness. From these case-studies we make four points: (1) ideal-typical regional innovation systems, i.e., regional clusters "surrounded" by supporting local organizations, is rather uncommon in Norway; (2) external contacts, outside of the local industrial milieu, are crucial in innovation processes also in many SMEs; (3) innovation processes may nevertheless be regarded as regional phenomena in regional clusters, as regional resources and collaborative networks often have decisive significance for firms' innovation activity; and (4) regional resources include in particular place-specific, contextual knowledge of both tacit and codified nature, that, in combination, is rather geographically immobile.
Autio, E. (1998). Evaluation of RTD in regional systems of innovation. European Planning Studies, 6(2), 131-140.
Abstract
This paper focuses on the evaluation of research and technical development (RTD) in regional systems of innovation (RSIs). It is argued that regional systems of innovation are distinctly different from national systems of innovation, and, thus, different approaches are called for in the evaluation of RSIs. The most relevant
 


aspects of RSIs, from the evaluation perspective, relate to their largely tacit and context-specific character. In this paper, the concept and characteristics of RSIs are reviewed, and the implications of these for evaluation practice are discussed. Pointers for good practice in the evaluation of RTD in RSIs are listed.
Breschi, S., & Malerba, F. (1997). Sectoral innovation systems: technological regimes, Schumpeterian dynamics, and spatial boundaries. In C. Edquist (Ed.), Systems of Innovation: Technologies, Institutions and Organizations (pp. 130-156).
London: Pinter Publishers.

Abstract
In this article, the concept of the sectoral innovation systems (SIS) is examined, in comparison with that of National Innovation Systems (NIS) and Technological Systems (TS). A sectoral innovation system can be defined as that system of firms active in developing and making a sector’s products and in generating and utilizing a sector’s technologies. Further, the authors claimed that the Technological Regimes (TR), defined by the level of opportunity and appropriability conditions, by the cumulativeness of technological knowledge, by the nature of knowledge and the means of knowledge transmission and communication, are a major factor that accounts for the dynamics of SISs and shape their spatial boundaries. Finally, an empirical analysis of some dimensions of SIS has been provided for six countries to confirm the relationship between TRs and SISs.
Burrus, D., and Gittines, R. (1993). Technotrends: how to use technology to go beyond your competition. New York: HarperBusiness.
Summary
The book was popular in the 1990s and deals with then-current and historical effects of technology in economic development. Among the key points of the book for innovation researchers and practitioners is the history of the continuously decreasing time between innovations and their commercialization. The implication of this fact and trend is the potential for marketing and profit-seeking
 


to overwhelm substance and fundamental importance in R&D and innovation processes. The book spurred the development of the firm Burrus Research, controlled by the author, who is now commonly referred to as a futurist and forecaster.
Bush, V. (1945). Science, the Endless Frontier: A Report to the President, from http://www.nsf.gov/od/lpa/nsf50/vbush1945.htm
Summary
In this post-WWII report to President Truman, Bush outlines a case for the United States to take on the role of funding basic research. Calling research a priority for disease prevention, public welfare and national security, Bush writes that scientific progress will improve the economic welfare of the nation. He also Congress to find ways to encourage more young people to go into scientific research and to strengthen patent laws to ensure that research is commercialized.
Carlsson, B., & et al. (2002). Innovation Systems: Analytical and Methodological Issues.
Research Policy, 31(2), 233-245.

Abstract
Innovation systems can be defined in a variety of ways: they can be national, regional, sectoral, or technological. They all involve the creation, diffusion, and use of knowledge. Systems consist of components, relationships among these, and their characteristics or attributes. The focus of this paper is on the analytical and methodological issues arising from various system concepts. There are three issues that stand out as problematic. First, what is the appropriate level of analysis for the purpose at hand? It matters, for example, whether we are interested in a certain technology, product, set of related products, a competence bloc, a particular cluster of activities or firms, or the science and technology base generally—and for what geographic area, as well as for what time period. The choice of components and system boundaries depends on this, as does the type of interaction among components to be analyzed. The attributes or features of the system components that come into focus also depend on the choice of level of analysis.
 


The second and closely related issue is how to determine the population, i.e. delineate the system and identify the actors and/or components. What are the key relationships that need to be captured so that the important interaction takes place within the system rather than outside? The third issue is how to measure the performance of the system. What is to be measured, and how can performance be measured at the system level rather than at component level?
Christopherson, S., and Clark, J. (2007). Power in firm networks: What it means for regional innovation systems. Regional Studies, 41(9), 1223-1236.
Abstract
The authors reject the commonly held notion that firms in regional networks cooperate to each other’s advantage. They argue that large firms, particularly transnational corporations (TNCs) use power, political and financial, to their own advantage. TNCs and other large firms work to restrict entry into their sectors and to shape R&D funding in their own interests.  Large enterprises also dominate labor supplies in their respective regions to the disadvantage of small and medium sized enterprises.
Colapinto, C. (2007). A way to foster innovation: a venture capital district from Silicon Valley and route 128 to Waterloo Region. International Review of Economics, 54(3), 319-343.
Abstract
This article is a comparative study between the two most traditionally cited US regions and a much smaller Canadian region. The author notes that Silicon Valley and the Route 128 corridor in the US are often cited as models to emulate. She argues that there are other smaller, less prominent examples that simply go unrecognized. Her comparative case is the Waterloo Region in Ontario, Canada. The article illustrates how the Canadian region shows strong inter-firm linkages, research communities, the financial sector, technology transfer from universities, combine with government agencies and government policy to grow and develop a software focused RIS in Canada.
 





Cook, G., Pandit, N., and Swann, G. (2001). The dynamics of industrial clustering in British broadcasting. Information Economics and Policy, 13(3), 351-375.
Abstract
The study attempts to generalize previously argued effects of industrial clustering in computing and biotech industries in the US and UK to a “non-high technology” service industry, radio and television broadcasting. The authors find support for cluster effects in television (except for ease of firm entry) but not radio (other than relative ease of firm entry). The article invites argument since the authors admit that satellite transmission and digital compression technology influence broadcasting, and include acting and actors as a principal component of labor input in broadcasting.
Cooke, P. (2001). Regional Innovation Systems, Clusters, and the Knowledge Economy.
Industrial and Corporate Change, 10(4), 945-974.

Abstract
This paper presents a systematic account of the idea and content of regional innovation systems following discoveries made by regional scientists, economic geographers and innovation analysts. It considers the conditions and criteria for empirical recognition and judgment as to whether scientifically analyzed, concrete cases of innovation activity warrant the designation of regional innovation system. The paper concludes by claiming that the source for Europe's innovation gap with the United States rests on excess reliance on public intervention, which signifies major market failure. The future will require widespread evolution of public innovation support systems along with stronger institutional and organizational support from the private sector.
Cooke, P., Gomez Uranga, M., & Etxebarria, G. (1997). Regional innovation systems: Institutional and organisational dimensions. Research Policy, 26(4-5), 475-491.
 


Abstract
The paper explores the case for Regional Systems of Innovation. Acknowledging the major contribution of research on National Innovation Systems, it suggests that for conceptual and methodological reasons, mostly concerning problems of scale and complexity, that approach may be complemented in important ways by a subnational focus. Taking an evolutionary economics standpoint, the paper specifies the concepts of 'region,' 'innovation' and 'system' as the prelude to an extended discussion of the importance of financial capacity, institutionalized learning and productive culture to systemic innovation. Building on the notion of regions as occupying different positions on a continuum referring to processes constituting them and their powers vis-à-vis innovation policy, the paper concludes by advocating strengthening of regional level capacities for promoting both systemic learning and interactive innovation.
Notes
This paper explains the logical and theoretical connections between NIS and RIS, thus to justify the merits for RIS in subnational innovation policy analysis.
Cooke, P., Heidenreich, M., & Braczyk, H. (2004). Regional Innovation Systems: The Role of Governance in a Globalized World. New York: Routledge.
Abstract
Set within a broadly evolutionary economics perspective, accounts are given of the systems interaction occurring between firms and the innovation support infrastructure. Case studies include 'high road' instances such as Baden- Württemberg, Brabant and Singapore, and reconversion regions which emphasize 'upstream' innovation such as Tampere (Finland) with close university-industry links or 'downstream' near-market innovation such as Catalonia. Policy implications of the analyses offered and variation explored are set in a context where regional administrations have limited access to the full scale of innovation policy instruments.
 


Notes
This book contains fourteen case studies which have been put into categories concerning three fundamental issues in the governance of RIS: local-global interaction, governance restructuring and interregional government cooperation.
Cooke, P., & Memedovic, O. (2003). Strategies for Regional Innovation Systems: Learning Transfer and Applications. Vienna, Austria: United Nations Industrial Development Organization.
Abstract
The paper explains the concept of regional innovation systems. It argues that global economic forces have raised the profile of regions and regional governance not least because of the rise to prominence of regional and local business clusters as vehicles for global and national economic competitiveness. Key definitions are given and distinctions drawn. Then, by reference to a number of important dimensions characterizing innovation such as education, knowledge transfer, linkage and communications, four regions from Asia, Europe and Latin America are contrasted. It is shown that regional innovation systems can be underdeveloped by being too dependent on public support, but equally, an over-emphasis on private infrastructures needs to be guarded against except at the most advanced developmental level.
Notes
A combination of public and private governance at regional level to promote systemic innovation is advocated.
Diez, J. R. (2002). Metropolitan innovation systems: A comparison between Barcelona, Stockholm, and Vienna. International Regional Science Review, 25(1), 63-85.
Abstract
This article uses data from the European Regional Innovation Survey to provide insights into the innovative activity and innovation networking of the most important innovation actors, namely manufacturing firms, producer service firms, and research institutes. The innovation capacities of the metropolitan innovation
 


systems differ markedly. In respect to cooperation partners, vertical relationships predominate. Only in Stockholm do research institutes play a significant role in assisting innovation processes in manufacturing firms. Spatial proximity of cooperation partners is very important, confirming the concept of territorially based systems of innovation. At the same time, the actors surveyed cooperate intensively with cooperation partners outside the region.
Notes
This paper identifies the key players in the metropolitan innovation system and compares the interaction models among them across three selected European regions.
Edquist, C. (1997). Systems of Innovation: Technologies, Institutions and Organizations.
London: Pinter Publishers.

Summary
Edquist has three main goals in editing this book: to define a systems approach to innovation research; to provide a conceptual framework for the systems approach connect that framework to current theory; and to examine how innovation is carried out and evolves over time. Edquist argues that the systems-based approach encompasses more than firms introducing new products. He writes that systems entail looking at the ways in which governments, nonprofit organizations and for- profit enterprises work together to create new knowledge.
Evangelista, R., & et al. (2002). Looking for Regional Systems of Innovation: Evidence from the Italian Innovation Survey. Regional Studies, 36(2), 173-186.
Abstract
The empirical target of this article is two-fold: exploring the variety of regional innovative patterns in Italy; and assessing whether innovation systems can be found, and how they operate, at a sub-national scale. The empirical analysis is based on an in-depth analysis of the data provided by the first Community Innovation Survey (CIS). The article shows that the traditional north-south distinction does not give full account of the wider spectrum of regional patterns in
 


Italy. In particular, regional innovative patterns differ not only according to the specific strategies and technological performances of firms, but also according to the relevance of systemic interactions and the presence of contextual factors favorable to innovation. However, proper regional systems of innovation are found only in a few well-defined areas. In most regions, systemic interactions and knowledge flows between the relevant actors are simply too sparse and too weak to reveal the presence of systems of innovation at work.
Notes
This paper provides a solution to identify and evaluate the RIS via tracking those key players innovation performance using survey information.
Fischer, M. M., Revilla Diez, J., & Snickars, F. (2001). Metropolitan innovation systems: Theory and evidence from three metropolitan regions in Europe. In association with Attila Varga. Advances in Spatial Science. Heidelberg and New York: Springer.
Summary
Presents a comparative study of the innovation systems of the Vienna, Barcelona, and Stockholm metropolitan areas. Identifies the main actors and mechanisms supporting technological innovation in each of the metropolitan regions based on responses to postal surveys sent to local manufacturing units, producer-service providers, and research institutions in each region. Compares and explains the similarities and differences in innovation systems of the selected metropolitan regions and sheds light on issues of innovation and networking activities, economic performance, and regional development. Presents policy implications for Europe's regions as they face new challenges associated with the emergence of a globalized knowledge-based economy. Fischer is at the Vienna University of Economics and Business Administration.
Freeman, C. (1987). Technology policy and economic performance: Lessons from Japan. London and New York: Pinter; distributed by Columbia University Press New York.
 


Summary
Concerned with innovation and its diffusion, following the Schumpeterian argument that technical and related social innovations are the main source of dynamism and instability in the world economy and that technical capacity is the main source of competitive strength of firms and nations. Develops the idea of a "national system of innovation" associated with pervasive technological changes. Focuses on the features of the Japanese system of innovations and their implications for other countries, concentrating on the institutions and experience of Japan. Begins with an international comparison of some long-term trends in science and technology indicators for the United States, Western Europe, and Japan, such as trends in research and development, gaps in productivity and technology, rates of growth, and output measures for science and technology.
Analyzes the Japanese national system of innovation. Features the role of the Ministry of International Trade and Industry, company research and development, education and training and social innovation, and the conglomerate structure of industry. Stresses the importance of information and communications and describes the Japanese system of technological forecasting and diffusion of major changes in technology throughout the economy. Indicates some of the problems for the world economy and Japan arising from the success of its technology policies, such as imbalances in world trade creating a world protectionist sentiment. The last Examines recent experiences in the United Kingdom in face of Japanese leadership and suggests programs for the U.S. and Europe.
Notes
Freeman's work here has been recognized widely as a breakthrough in understanding the sources and mechanisms of innovation with a systemic manner. Observations draw mainly on the Japanese case.
Freeman, C. (2002). Continental, national and sub-national innovation systems— complementarity and economic growth. Research Policy, 31(2), 191-211.
 


Abstract
Freeman addresses two crucial issues which emerged in the 1990s as a result of the “rapidly growing literature” around the concept of innovation systems. First, the word “regional” leads to confusion in combination with innovation systems. The literature describes nation states, sub-continents, and supra-national organizations as regions, and uses the same word to describe sub-national states and provinces, metropolitan areas, cities, counties and rural areas. Second Freeman takes issue with classical economists who have argued historically and recently that economic growth can be explained by the accumulation of (physical or financial) capital and an increased labor supply. Freeman argues for an institutional approach to describe “social capacity” to adopt and adapt to new technology and skills.
Freeman, C., and Soete, L. (2009). Developing science, technology and innovation indicators: what we can learn from the past. Research Policy, 38(4), 583-589.
Abstract
The authors take a critical and historical perspective on the potential for misuse of traditional indicators of science and technology innovation (STI). Some indicators may be adopted for their ease of measurability, leading to threats of validity.
National economies can vary dramatically from agriculture to manufacturing to services, rendering indicators not directly comparable. Knowledge leakage and spillovers may make it difficult to capture the value of indicators where they were generated. The article focuses on national and global economies with no reference to sub-national regions.
Gertler, M., Wolfe, D., & Garkut, D. (1998). The dynamics of regional innovation in Ontario. In J. de la Mothe & G. Paquet (Eds.), Local and Regional Systems of Innovation (pp. 211-238). New York: Springer-Verlag.
Summary
The authors use an innovation survey of firms in Ontario, Canada, to study both internal innovation and innovation through network relations. Gertler, et. al., write
 


that the modern technology economies will require firms to join together in order to gain competitive advantage through technological innovation. They conclude that Ontario is not forming a densely networked economy, citing the province's regulatory environment, decentralized labor market and short-term focus of the capital markets as mitigating against the formation of mutually cooperative firms. But the authors note that Ontario is responding to globalization by taking advantage of the North American Free Trade Agreement and firms are doing more R&D.
Griliches, Z. (1990). Patent Statistics as Economic Indicators - A Survey. Journal of Economic Literature, 28(4), 1661-1707.
Summary
Griliches writes that there are two major problems with patents: classification and intrinsic variability. Classification into different industries, he argues, is largely a technical issue. But patents also have a lot of variability in the quality of the new innovation. He writes that there is a strong relationship between patents and R&D expenditures at firms, so it can be used as an indicator of inventive activity across firms. The author describes other uses of the patent data, such as seeing how patents spill over into new innovations at other firms. Using the data on a macroeconomic level is not as useful.
Grupp, H., & Mogee, M. E. (2004). Indicators for national science and technology policy: How robust are composite indicators? Research Policy, 33(9), 1373-1384.
Abstract
This article addresses a set of issues that were central to Keith Pavitt's research. that is the construction and use of tools to measure national innovative performance and to design national policies relating to innovation. It presents an overview of the development of science and technology (S&T) indicators and their use in national policy making and provides evidence of the vulnerability of composite S&T indicators to manipulation. A brief history of the development of S&T indicators begins with the role of the United States followed by their
 


worldwide diffusion with particular emphasis on Europe. Newer developments towards composite indicators, benchmarking and scoreboarding are discussed. To investigate the robustness of innovation scoreboards, empirically, a sensitivity analysis of one selected case is presented. It is shown that composite scores and country rank positions can vary considerably depending on the selection process. Thus, the use of scoreboards leaves room for manipulation in the policymaking system. Further research is needed on alternative methods of calculation to prevent their misuse and abuse.
Hall, J. L. (2007). Developing historical 50-state indices of innovation capacity and commercialization capacity. Economic Development Quarterly, 21(2), 107-123.
Abstract
Recent attention to innovation as the core of a knowledge-based economy has resulted in an array of studies and reports that seek to measure states' relative ranks as they advance their economic agendas. This study improves on state performance measurement by distinguishing innovation capacity from innovation outcomes by examining change over a 20-year period with consistent measures and by empirically grouping measures into core resource categories using factor analysis. Factor analysis is used to generate new measures of innovation capacity, and the efficacy of these new measures is tested using pooled cross-sectional time- series analysis to examine their effects on state patent generation. The findings indicate moderate to strong impacts of the innovation capacity variables on patent generation; the results provide a new grounded metric for examining state capacity for innovation and state financial capacity for commercialization over time.
Hall, J. L. (2009). Adding Meaning to Measurement Evaluating Trends and Differences in Innovation Capacity among the States. Economic Development Quarterly, 23(1), 3-12.
Abstract
How do states compare to one another, and to themselves, in innovation capacity and past innovation performance? Are there groups of states that are more or less
 


similar in innovation capacity composition? Because different score dimensions vary independently, it is possible for states to be high on some dimensions and low on others. In an effort to give greater meaning to innovation index scores, it is necessary to evaluate the relationships among them. This article subjects Hall's innovation capacity index scores to cluster analysis to reveal clusters of states that are similar in innovation capacity levels across the three dimensions considered. A cluster typology is created, and state changes in typology are observed and compared over the 20-year period of the data set. Patterns observed across states and over time will help policy makers to identify major changes in their typology that may reflect goal progress or regression.
Holbrook, A., & Salazar, M. (2004). Regional Innovation Systems within A Federation: Do national policies affect all regions equally? Innovation: Management, Policy & Practice, 6(1), 50-64.
Abstract
The concept of national innovation systems was first developed to describe the process of innovation in developed economies. The approach has shifted from solely a national perspective to one including regional or local systems. This focus on spatial aspects has two major advantages: it recognizes that innovation is a social process and a geographic process. For federations, the national system of innovation is more complex than that of a unitary system, since there are often provincial/state level institutions and actors that parallel national level institutions and actors. Canada is one of the few true economic and social (as well as political) federations in the developed world. Consequently, it provides a unique laboratory for studies on the processes of innovation in regions and regional innovation systems. This paper reports on the initial results of research on the characteristics of industrial clusters being carried out through the (Canadian) Innovation Systems Research Network - ISRN.
 


Isaksen, A. (2001). Building Regional Innovation Systems: Is Endogenous Industrial Development Possible in the Global Economy? Canadian Journal of Regional Science, 24(1), 101-120.
Abstract
The article discusses regionalization as an important aspect of economic globalization and as a starting point in shaping endogenous industrial policy that is adapted to specific regional circumstances. For these tasks, the article suggests definitions of central concepts as regional clusters, regional innovations systems and systems barriers that emphasis the importance of "non-economic" factors to a much larger extent than typically found in the Porterian approach. The article then refers to the a consolidation attempt on the part of Ericsson, which took place in Norway a few years ago, in order to illustrate both threats and possibilities for local industrial development in the global economy. This event includes the decision made by the transnational corporation Ericsson to relocate one of their development departments from a small Norwegian town to the capital region, and the later change of plan because very few of the engineers seem to be willing to move along with the department. Lastly, the article departs from the Ericsson event to discuss, from the regional innovation system perspective, possible development policies to anchor units of transnational corporations to a local area.
Kleinknecht, A., van Montfort, K., & Brouwer, E. (2002). The Non-trivial Choice between Innovation Indicators. Economics of Innovation and New Technology, 11(2), 109-121.
Abstract
We discuss the strengths and weaknesses of five alternative innovation indicators: R&D, patent applications, total innovation expenditure and shares in sales taken by imitative and by innovative products as they were measured in the 1992 Community Innovation Survey (CIS) in the Netherlands. We conclude that the two most commonly used indicators (R&D and patent applications) have more (and more severe) weaknesses than is often assumed. Moreover, our factor analysis
 


suggests that there is little correlation between the various indicators. This underlines the empirical relevance of various sources of bias of innovation indicators as discussed in this paper.
Kline, S. J., & Rosenberg, N. (1986). An Overview of Innovation. In R. Landau & N. Rosenberg (Eds.), The positive sum strategy: Harnessing technology for economic growth (pp. 275-305). Washington, D. C.: National Academy Press.
Abstract
Models that depict innovation as a smooth, well-behaved linear process badly mis- specify the nature and direction of the causal factors at work. Innovation is complex, uncertain, somewhat disorderly, and subject to changes of many sorts.
Innovation is also difficult to measure and demands close coordination of adequate technical knowledge and excellent market judgment in order to satisfy economic, technological, and other types of constraints - all simultaneously. The process of innovation must be viewed as a series of changes in a complete system not only of hardware, but also of market environment, production facilities and knowledge, and the social contexts of the innovation organization.
Krauss, G., and Wolf, H. (2002). Technological Strengths in Mature Sectors--An Impediment or an Asset for Regional Economic Restructuring? The Case of Multimedia and Biotechnology in Baden-Wurttemberg. The Journal of Technology Transfer, 27(1), 39-50.
Abstract
The authors study obstacles to developing new high technology industries (multimedia and biotechnology) in the Baden-Wurttemberg region. The region’s innovation system is characterized by mature industries including automobiles, machinery, electronics and electrical engineering. They conclude that multimedia firms share many characteristics with their existing sectors. Biotechnology does not. Since major firms, labor and other actors in the region are institutionalized and deeply embedded it will be difficult to develop a biotechnology sector successfully. They generalize that finding.  That is, bringing new industry sectors
 


into an innovation system is risky if the new sector does not share technology, skills and knowledge that are already institutionalized in the local region.
Liu, S., and Chen, C. (2003). Regional innovation system: Theoretical approach and empirical study of China. Chinese Geographical Science, 13(3), 193-198.
Abstract
The authors develop their conceptual definition of regional innovation systems and apply it to multi-provincial area in China, including major cities and an autonomous region. They conclude that there are major differences in scale and quality of regional innovation outcomes, attributable to diverse social and economic conditions in different regions within China. China, like the US, is noteworthy for its vast geography unlike the individual states in Europe.
Lundvall, B.-A. (1992). National systems of innovation: Towards a theory of innovation and interactive learning. London: Pinter; distributed in the U.S. and Canada by St. Martin's Press New York.
Abstract
Thirteen papers combine the French structuralist approach to national systems of production and the Anglo-Saxon tradition in innovation studies in order to explain international competitiveness. Papers focus on a new approach to national systems of innovation; a closer look at national systems of innovation; and specialization, multinational corporations, and integration.
Malecki, E. (1997). Technology and economic development: The dynamics of local, regional and national competitiveness. Harlow: Addison Wesley Longman.
Summary
This book is updated from Malecki’s 1991 edition. The book is slightly dated but provides an excellent historic overview of innovation-driven economic growth and development. Malecki includes an important definition of technology as applied knowledge, separated from more recent tendencies to link or equate technology with electronics, information technology and telecommunications, or software.
 


The author dates the period of innovation-driven development from the late 1700s, linked to the emergence of formal capitalist economic systems. The book is useful for practitioners and researchers, and should be included in any comprehensive study of regional innovation.
Malerba, F. (2004). Sectoral systems of innovation: Concepts, issues and analyses of six major sectors in Europe. Cambridge; New York and Melbourne: Cambridge University Press.
Abstract
Twelve papers apply a sectoral systems of innovation framework to analyze innovation in some major sectors in Europe. Papers discuss sectoral systems of innovation and production and their main building blocks; sectoral dynamics and structural change; pharmaceuticals analyzed through the lens of a sectoral innovation system; the processes of knowledge creation and diffusion in the chemical sectoral system; the fixed Internet and mobile telecommunications sectoral system of innovation; the European software sectoral system of innovation; the remaking of innovation processes and boundaries in the machine tool industry; services and systems of innovation; the role of institutions in sectoral systems of innovation; the interplay between national institutional frameworks and sectoral specialization; the factors affecting the international performance of European sectoral systems; and implications for European innovation policy.
Nelson, R. R. (1959). The Simple Economics of Basic Scientific Research. The Journal of Political Economy, 67(3), 297-306.
Abstract
Nelson argues that basic scientific research provides a wide range of positive economic externalities, but it is not easily privatized, because the research benefits a variety of different fields and firms. Predominantly research is not conducted by industry because it is costly and may not provide a benefit to the firm. Also, basic research at firms is economically inefficient, because the knowledge will not be
 


used by a wide range of researchers. Since there are costs to private industry for basic research, he suggests that the evidence suggests that government should provide more support to take the burden of that research off the hands of private industry.
Nelson, R. R. (1993). National innovation systems: A comparative analysis. Oxford; New York; Toronto and Melbourne: Oxford University Press.

Abstract
Fourteen papers examine national systems of technical innovation in fifteen countries. Studies are designed, developed, and written to illuminate the institutions and mechanisms supporting technical innovation in the various countries, the similarities and differences across countries and how these came to be, and how the differences matter. Countries discussed are the United States, Japan, Germany, the United Kingdom, France, Italy, Denmark, Sweden, Canada, Australia, South Korea, Taiwan, Brazil, Argentina, and Israel.
NSF (1956). Expenditures for R&D in the United States 1953. Washington, D.C.: National Science Foundation.
Article unavailable.

NSF (1957). Basic Research: A national resource. Washington, D.C.: National Science Foundation.
Article unavailable.

OECD (1963). Proposed Standard Practice for Surveys of Research and Development.
Paris: Directorate for Scientific Affairs. OECD.

Summary

As the internationally recognized methodology for collecting and using R&D statistics, this chapter is an essential tool for statisticians worldwide. It includes definitions of basic concepts, data collection guidelines, and classification for compiling statistics.
 


OECD (1992). Oslo Manual: Proposed Guidelines for Collecting and Interpreting Technological Innovation Data: Organisation for Economic Cooperation and Development.
Article unavailable.

OECD (1997). National Innovation Systems: Organisation for Economic Cooperation and Development.
Abstract
Systemic approaches are giving new insight to innovative and economic performance in OECD countries. The interactions among the firms, institutions and others involved in technology development are now seen to be as important as direct investment in R&D. This publication discusses the first phase of OECD work on national innovation systems and the attempt to develop indicators to map knowledge flows.
OECD, & Office, E. C. S. (1997). Oslo Manual: Proposed Guidelines for Collecting and Interpreting Technological Innovation Data: OECD/Eurostat.
Summary
This manual summarizes the definitions, criteria and methodologies for the studies of industrial innovation, the operation of international and national innovation surveys and the choice of indicators. Some alternative approaches other than those that had been included in the Community Innovation Survey (CIS) are also provided in contrasts to the 1992 version Oslo Manual.
OECD, & Office, E. C. S. (2005). Oslo Manual: OECD/Eurostat.

Summary
The Oslo Manual outlines a framework for conducting research into innovation. Responding to recent literature on a systems approach to innovation, the most recent edition introduces a chapter on innovation linkages. It also introduces two new types of innovation: marketing - changes in packaging or pricing - and
 


organizational - changes in business practices. These innovation measures better define innovation in the service sector.
Oinas, P., & Malecki, E. J. (2002). The evolution of technologies in time and space: From national and regional to spatial innovation systems. International Regional Science Review, 25(1), 102-131.
Abstract
Complementing existing approaches on national innovation systems (NISs) and regional innovation systems (RISs), the proposed spatial innovation systems (SISs) approach incorporates a focus on the path-dependent evolution of specific technologies as components of technological systems and the intermingling of
their technological paths among various locations through time. SISs utilize spatial divisions of labor among several specialized RISs, possibly in more than one NIS. The SIS concept emphasizes the external relations of actors as key elements that transcend all existing systems of innovation. The integrating role of these relations remains inadequately understood to date. This poses a challenge for future research.
Pavitt, K., Robson, M., & Townsend, J. (1987). The Size Distribution of Innovating Firms in the UK - 1945-1983. Journal of Industrial Economics, 35(3), 297-316.
Abstract
A survey of 4378 significant innovations shows that firms with fewer than 1000 employees commercialized a much larger share than is indicated by their share of R&D expenditures. Innovations per employee have been consistently above average in firms with more than 10000 employees, and have become so in firms with fewer than 1000. Intersectoral variation in the size distribution of innovating firms can be explained as a function of R&D-based technological opportunities, and of "technological ease of entry" by user firms with principal activities outside the sector.
 


Porter, M., & Stern, S. (1999). The New Challenge to America's Prosperity: Findings from the Innovation Index (No. 1-889866-21-0). Washington, D.C.: Council on Competitiveness.
Summary
This report by the Council on Competitiveness tracks the relative innovation capacities of 17 OECD economies and eight emerging economies using an Innovation Index developed by the authors. The rankings show that the United States could lose its leadership role in innovation because of declining commitment to innovation. The report identifies three main areas of innovative capacity: a common innovation infrastructure, cluster-specific conditions, and the strength of the linkages among them. It also gives the methodology for creating the index, and ways to weight different factors for regression analysis, which will be useful for RRI’s regional innovation study.
Rosenberg, N. (1982). Inside the Black Box: Technology and Economics. New York: Cambridge University Press.
Notes
This book is among those classics that examine the relationship between technology progress and economic development and the economic, political, social and cultural determinants of technology progress.
Sajeva, M., & Gatelli, D. (2005). Methodology Report on European Innovation Scoreboard 2005: European Commission, Enterprise Directorate-General.
Summary
The authors find that changes to indicators and methodology of the European Innovation Scoreboard (EIS) in 2005 did not markedly change the robustness of the results. The findings recommend equal weighting of indicators, and no imputation to fill in missing data.
Scott, A. (2006). Entrepreneurship, innovation and industrial development: geography and the creative field revisited. Small Business Economics, 26(1), 1-24.
 


Abstract
The author examines the “creative field” within the “cultural products” industry. Examples of cultural products include motion pictures, music, electronic games, architecture, tourism and the fashions. The creative filed is characterized by numerous and diverse small firms and individual entrepreneurs such as writers, performers, designers, and graphic artists. The larger industry also includes large firms with capital resources necessary to make movies, publish books and record and distribute music. Dense inter-linkages run vertically and horizontally within the industry and its sub-fields. Thus the “geographic expression” is typically in dense urban areas such as New York, Paris, Los Angeles, Tokyo and London.
Simmie, J. (2003). Innovation and urban regions as national and international nodes for the transfer and sharing of knowledge. Regional Studies, 37(6-7), 607-620.
Abstract
This paper examines the transfer and sharing of knowledge within and between regions in the context of the development of the international economy. It is argued that knowledge is a key resource for innovation which, in turn, is one of the major drivers of economic growth. The firms producing the most novel product innovations in the most significant regional concentrations of innovation are very adept at working across the interface of local and global knowledge transfers. Using data from previous studies combined with the latest regional data from the Community Innovation Survey 3, comparisons are made between the ways in which the most innovative firms in the Greater South East transfer and share knowledge from the local to the international level. The most innovative firms are shown to access international sources of knowledge. This raises questions over the relative importance of local versus international knowledge spillovers for the most innovative firms. Innovative firms tend to concentrate in a minority of key metropolitan regions. These are shown to combine a strong local knowledge capital base with high levels of connectivity to similar regions in the international economy. In this way they are able to combine and decode both codified and tacit knowledge originating from multiple regional, national and
 


international sources. As a result they are able to generate virtuous circles of knowledge, innovation, competitiveness and exports.
Smith, K. (2005). Measuring Innovation. In J. Fagerberg, D. C. Mowery & R. R. Nelson (Eds.), The Oxford handbook of innovation (pp. 148-177). Oxford and New York: Oxford University Press.
Abstract
It is sometimes suggested that innovation is inherently impossible to quantify and to measure. This article argues that while this is true for some aspects of innovation, its overall characteristics do not preclude measurement of key dimensions of processes and outputs. An important development has been the emergence of new indicators of innovation inputs and outputs. Following sections discuss first some broad issues in the construction and use of science, technology and innovation indicators, then turn briefly to the strengths and weaknesses of current indicators particularly R&D and patents. Final sections cover recent initiatives focusing on the conceptualization, collection, and analysis of direct measures of innovation, especially the rapidly growing use of the Community Innovation Survey (CIS).
Soete, L. (2006). Knowledge, policy and innovation. In L. Earl & F. Gault (Eds.), National Innovation, Indicators and Policy (pp. 198-218). Cheltenham: Edward Elgar.
Summary
In this survey of available literature, Soete concludes that the traditional definitions of research and development need to be expanded. He writes that countries with high research capacity do not necessarily have high economic growth if they do not have the proper institutional context to allow innovation to thrive. He concludes that four factors are crucial for innovation: social and human capital; research capacity; geographical proximity; and absorptive capacity. All four should be encouraged by policy makers.
 


Soete, L. (2007). From industrial to innovation policy. Journal of Industry, Competition and Trade, 7, 273-284.
Abstract
This article is a historical discussion of how Europe moved from an industrial policy focused on coal and steel as the basis of European integration after World War II to a more innovative policy of innovation aimed at fostering high technologies that also serve social and environmental goals, such as pollution control and “clean” technology. Soete discusses Europe as an example of national, sub-national and supra national regions combined, where externalities
development choices easily cross borders. The article is easily accessible by practitioners and may resonate in sub-national regions of the US characterized by industries such as mining and agriculture. Researchers who are not familiar with recent European history and arguments about appropriate industrial policies will find the article interesting as well.
Stokes, D. E. (1997). Pasteur's quadrant: Basic science and technological innovation.
Washington, D.C.: Brookings Institution Press.

Abstract
Proposes a revised view of the relationship between basic science and technological innovation and shows how this revision could lead to a clearer view of several aspects of science and technology policy. Describes the problematic aspects of the postwar paradigm that basic science can serve as a pacemaker of technological progress only if it is insulated from thought of practical use.
Addresses the paradox of how this vision of science and its role in
technological innovation could have prevailed, given that those who
built modern science were so often influenced by applied goals. Sets
out a more realistic view of the links between basic science and
technological innovation that is more faithful to the history of
research. Considers renewing the compact between science and
government. Considers a process by which American democracy could
build agendas of use-inspired basic research by bringing together judgments of research promise and societal need.
Tijssen, R. J. W. (2003). Scoreboards of research excellence. Research Evaluation, 12(2), 91-104
Abstract
A critical discussion is presented of what could be understood as research excellence, and how to deal with fundamental issues and methodological challenges in operationalizing and evaluating this complex, multi-faceted notion in terms of measurable attributes at organizational levels. This paper argues for a systemic and interactive approach, combining multiple perspectives and stakeholders, while incorporating a wide range of information sources and quantitative indicators within the analytical framework of a 'scoreboard'. Context- specific and customized scoreboards show promise as a structuring tool in informed debate, indicator selection, comparative analysis and benchmarking studies of research excellence. Guidelines and recommendations are illustrated by way of a fictitious scoreboard with recent empirical data for economics research at the universities in the Netherlands.



\bibliography{regional_innovation}
\bibliographystyle{plainnat}

\end{document}