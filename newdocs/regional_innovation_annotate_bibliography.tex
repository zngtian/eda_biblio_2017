% Created 2017-08-22 Tue 15:29
% -*- latex-run-command: xelatex -*-
\documentclass[a4paper,11pt]{article}
\usepackage{graphicx}
\usepackage{grffile}
\usepackage{longtable}
\usepackage{wrapfig}
\usepackage{rotating}
\usepackage[normalem]{ulem}
\usepackage{amsmath}
\usepackage{textcomp}
\usepackage{amssymb}
\usepackage{capt-of}
\usepackage{hyperref}
\usepackage[margin=1in]{geometry}
\usepackage{setspace}
\onehalfspacing
\usepackage{parskip}
\usepackage{tabularx}
\usepackage{color}
\usepackage{caption}
\usepackage{subcaption}
\usepackage[round]{natbib}
\hypersetup{colorlinks,citecolor=black,filecolor=black,linkcolor=black,urlcolor=black}
\setcounter{secnumdepth}{2}
\author{Zheng Tian, Jing Chen}
\date{\today}
\title{Annotated Bibliography (Updated)}
\hypersetup{
 pdfauthor={Zheng Tian, Jing Chen},
 pdftitle={Annotated Bibliography (Updated)},
 pdfkeywords={},
 pdfsubject={},
 pdfcreator={Emacs 25.1.1 (Org mode 9.0.9)}, 
 pdflang={English}}
\begin{document}

\maketitle


\section{Introduction}
 \label{introduction}

\section{Literature by Category}
\label{lit_category}

\subsection{General Theory}

\begin{enumerate}
\item Breschi, S., \& Malerba, F. (1997). Sectoral Innovation Systems: Technological Regimes, Schumpeterian Dynamics and Spatial Boundaries. In C. Edquist (Ed.), Systems of Innovation: Technologies, Institutions and Organizations (pp. 130–156). Pinter Publishers.
\item Burrus, D., \& Gittines, R. (1993). Technotrends: How to Use Technology to Go Beyond Your Competition (1st edition). New York: Harpercollins.
\item Bush, V. (1960). Science, the endless frontier: a report to the President on a program for postwar scientific research.
\item Carlsson, B., Jacobsson, S., Holmén, M., \& Rickne, A. (2002). Innovation systems: analytical and methodological issues. Research Policy, 31(2), 233–245. https://doi.org/10.1016/S0048-7333(01)00138-X
\item Edquist, C. (Ed.). (2012). Systems of Innovation: Technologies, Institutions and Organizations (1 edition). London: Routledge.
\item Freeman, C. (1987). Technology Policy and Economic Performance: Lessons from Japan. London ; New York: Pinter Pub Ltd.
\item Freeman, C., \& Soete, L. (2009). Developing science, technology and innovation indicators: What we can learn from the past. Research Policy, 38(4), 583–589. https://doi.org/10.1016/j.respol.2009.01.018
\item Kline, S. J., \& Rosenberg, N. (1986). An Overview of Innovation. In R. Landau \& N. Rosenberg (Eds.), The positive sum strategy: Harnessing technology for economic growth (pp. 275–305). Washington, D.C.
\item Lundvall, B.-Å. (Ed.). (2010). National Systems of Innovation: Toward a Theory of Innovation and Interactive Learning (Revised ed. edition). London: Anthem Press.
\item Malecki, E. J. (1997). Technology and Economic Development: The Dynamics of Local, Regional and National Competitiveness (2 edition). Essex, England: Longman Pub Group.
\item Malerba, F. (Ed.). (2004). Sectoral Systems of Innovation: Concepts, Issues and Analyses of Six Major Sectors in Europe. New York, N.Y: Cambridge University Press.
\item National Science Foundation (U.S.). (1957). Basic Research: A National Resource. Washington, D.C.: National Science Foundation. Retrieved from https//catalog.hathitrust.org/Record/006685493
\item Nelson, R. R. (1959). The Simple Economics of Basic Scientific Research. Journal of Political Economy, 67, 297–297.
\item Nelson, R. R. (Ed.). (1993). National Innovation Systems: A Comparative Analysis (1 edition). New York: Oxford University Press.
\item Oinas, P., \& Malecki, E. J. (2002). The Evolution of Technologies in Time and Space: From National and Regional to Spatial Innovation Systems. International Regional Science Review, 25(1), 102–131. https://doi.org/10.1177/016001702762039402
\item Rosenberg, N. (1983). Inside the Black Box: Technology and Economics. Cambridge Cambridgeshire ; New York: Cambridge University Press.
\item Soete, L. (2007). From Industrial to Innovation Policy. Journal of Industry, Competition and Trade, 7(3–4), 273. https://doi.org/10.1007/s10842-007-0019-5
\item Stokes, D. E. (1997). Pasteur’s Quadrant: Basic Science and Technological Innovation. Washington, D.C: Brookings Institution Press.
\end{enumerate}

\subsection{Regional Innovation}

\begin{enumerate}
\item Alcaide-Marzal, J., \& Tortajada-Esparza, E. (2007). Innovation assessment in traditional industries. A proposal of aesthetic innovation indicators. Scientometrics, 72(1), 33–57. https://doi.org/10.1007/s11192-007-1708-x
\item Asheim, B. T., \& Isaksen, A. (2002). Regional Innovation Systems: The Integration of Local ‘Sticky’ and Global ‘Ubiquitous’ Knowledge. The Journal of Technology Transfer, 27(1), 77–86. https://doi.org/10.1023/A:1013100704794
\item Autio, E. (1998). Evaluation of RTD in regional systems of innovation. European Planning Studies, 6(2), 131–140. https://doi.org/10.1080/09654319808720451
\item Carlsson, B., Jacobsson, S., Holmén, M., \& Rickne, A. (2002). Innovation systems: analytical and methodological issues. Research Policy, 31(2), 233–245. https://doi.org/10.1016/S0048-7333(01)00138-X
\item Christopherson, S., \& Clark, J. (2007). Power in Firm Networks: What it Means for Regional Innovation Systems. Regional Studies, 41(9), 1223–1236. https://doi.org/10.1080/00343400701543330
\item Colapinto, C. (2007). A way to foster innovation: a venture capital district from Silicon Valley and route 128 to Waterloo Region. International Review of Economics, 54(3), 319–343. https://doi.org/10.1007/s12232-007-0018-1
\item Cooke, P. (2001). Regional Innovation Systems, Clusters, and the Knowledge Economy. Industrial and Corporate Change, 10(4), 945–974. https://doi.org/10.1093/icc/10.4.945
\item Cooke, P., Gomez Uranga, M., \& Etxebarria, G. (1997). Regional innovation systems: Institutional and organisational dimensions. Research Policy, 26(4), 475–491. https://doi.org/10.1016/S0048-7333(97)00025-5
\item Cooke, P. N., Heidenreich, M., \& Braczyk, H.-J. (2004). Regional Innovation Systems: The Role of Governance in a Globalized World. Psychology Press.
\item Evangelista, R., Iammarino, S., Mastrostefano, V., \& Silvani, A. (2002). Looking for Regional Systems of Innovation: Evidence from the Italian Innovation Survey. Regional Studies, 36(2), 173–186. https://doi.org/10.1080/00343400220121963
\item Freeman, C. (2002). Continental, national and sub-national innovation systems—complementarity and economic growth. Research Policy, 31(2), 191–211. https://doi.org/10.1016/S0048-7333(01)00136-6
\item Krauss, G., \& Wolf, H.-G. (2002). Technological Strengths in Mature Sectors--An Impediment or an Asset for Regional Economic Restructuring? The Case of Multimedia and Biotechnology in Baden-Wurttemberg. The Journal of Technology Transfer, 27(1), 39–50. https://doi.org/10.1023/A:1013144519815
\item Liu, S., \& Chen, C. (2003). Regional innovation system: Theoretical approach and empirical study of China. Chinese Geographical Science, 13(3), 193–198. https://doi.org/10.1007/s11769-003-0016-5
\item Scott, A. J. (2006). Entrepreneurship, Innovation and Industrial Development: Geography and the Creative Field Revisited. Small Business Economics, 26(1), 1–24. https://doi.org/10.1007/s11187-004-6493-9
\item Simmie, J. (2003). Innovation and Urban Regions as National and International Nodes for the Transfer and Sharing of Knowledge. Regional Studies, 37(6–7), 607–620. https://doi.org/10.1080/0034340032000108714
\end{enumerate}

\subsection{Methodology}

\begin{enumerate}
\item Acs, Z. J., Anselin, L., \& Varga, A. (2002). Patents and innovation counts as measures of regional production of new knowledge. Research Policy, 31(7), 1069–1085.
\item Acs, Z. J., \& Audretsch, D. B. (1993). Analysing Innovation Output Indicators: The US Experience. In A. Kleinknecht \& D. Bain (Eds.), New concepts in innovation output measurement (pp. 10–41). Palgrave Macmillan UK. https://doi.org/10.1007/978-1-349-22892-8-2
\item Arundel, A. (2007). Innovation Survey Indicators: What Impact on Innovation Policy? In D. Organisation for Economic Co-operation and (Ed.), Science, Technology and Innovation Indicators in a Changing World: Responding to Policy Needs.
\item Edquist, C. (1997). Systems of innovation: technologies, institutions, and organizations. Routledge. https://doi.org/10.4324/9780203357620
\item Evangelista, R., \& et al. (2002). Looking for Regional Systems of Innovation: Evidence from the Italian Innovation Survey. Regional Studies, 36(2), 173–186.
\item Gertler, M. S., Wolfe, D. A., \& Garkut, D. (1998). The Dynamics of Regional Innovation in Ontario. In Local and Regional Systems of Innovation (pp. 211–238). Boston, MA: Springer, Boston, MA. https://doi.org/10.1007/978-1-4615-5551-3-11
\item Griliches, Z. (1990). Patent Statistics as Economic Indicators - A Survey. Journal of Economic Literature, 28(4), 1661–1707.
\item Grupp, H., \& Mogee, M. E. (2004). Indicators for national science and technology policy: How robust are composite indicators? Research Policy, 33(9), 1373–1384. https://doi.org/10.1016/j.respol.2004.09.007
\item Hall, J. L. (2008). Adding Meaning to Measurement. Economic Development Quarterly, 23(1), 3–12. https://doi.org/10.1177/0891242408326467
\item Hall, J. L. (2016). Developing Historical 50-State Indices of Innovation Capacity and Commercialization Capacity. Economic Development Quarterly, 21(2), 107–123. https://doi.org/10.1177/0891242406298128
\item Kleinknecht, A., \& Van Montfort, K. (2002). The non-trivial choice between innovation indicators. Economics of Innovation, 11(2), 109–121. https://doi.org/10.1080/10438590210899
\item NSF. (1956). Expenditures for R\&D in the United States 1953. Washington, D.C.: National Science Foundation.
\item OECD. (1963). Proposed Standard Practice for Surveys of Research and Development. Paris: Directorate for Scientific Affairs. OECD.
\item OECD. (1992). Oslo Manual: Proposed Guidelines for Collecting and Interpreting Technological Innovation Data.
\item OECD. (1997). National Innovation Systems. Organization for Economic Cooperation and Development.
\item OECD, \& European Communities Statistical Office. (2005). Oslo Manual: Proposed Guidelines for Collecting and Interpreting Technological Innovation Data. OECD/Eurostat.
\item Porter, M., \& Stern, S. (1999). The New Challenge to America’s Prosperity: Findings from the Innovation Index. Washington, D.C.: Council on Competitiveness.
\item Sajeva, M., \& Gatelli, D. (2005). Methodology Report on European Innovation Scoreboard 2005. European Commission, Enterprise Directorate-General.
\item Simmie, J. (2003). Innovation and urban regions as national and international nodes for the transfer and sharing of knowledge. Regional Studies, 37(6–7), 607–620. https://doi.org/10.1080/0034340032000108714
\item Smith, K. H. (2005). Measuring innovation. In J. Fagerberg \& D. C. Mowery (Eds.), The Oxford Handbook of Innovation. Oxford University Press. https://doi.org/10.1093/oxfordhb/9780199286805.003.0006
\item Tijssen, R. (2003). Scoreboards of research excellence. Research Evaluation, 12(2), 91–103. https://doi.org/10.3152/147154403781776690
\end{enumerate}

\subsection{Application}

\begin{enumerate}
\item Arundel, A. (2007). Innovation Survey Indicators: What Impact on Innovation Policy? In D. Organisation for Economic Co-operation and (Ed.), Science, Technology and Innovation Indicators in a Changing World: Responding to Policy Needs.
\item Asheim, B. T., \& Isaksen, A. (2002). Regional Innovation Systems: The Integration of Local “Sticky” and Global “Ubiquitous” Knowledge. Journal of Technology Transfer, 27(1), 77–86.
\item Cooke, P., Heidenreich, M., \& Braczyk, H. J. (2004). Regional Innovation Systems: The Role of Governance in a Globalized World. New York: Routledge.
\item Cooke, P., \& Memedovic, O. (2003). Strategies for Regional Innovation Systems: Learning Transfer and Applications. Vienna, Austria: United Nations Industrial Development Organization.
\item Diez, J. R. (2002). Metropolitan innovation systems: A comparison between Barcelona, Stockholm, and Vienna. International Regional Science Review, 25(1), 63–85.
\item Evangelista, R., \& et al. (2002). Looking for Regional Systems of Innovation: Evidence from the Italian Innovation Survey. Regional Studies, 36(2), 173–186.
\item Fischer, M. M., Revilla Diez, J., \& Snickars, F. (2001). Metropolitan innovation systems: Theory and evidence from three metropolitan regions in Europe. In association with Attila Varga. Advances in Spatial Science. Heidelberg and New York: Springer.
\item Grupp, H., \& Mogee, M. E. (2004). Indicators for national science and technology policy: How robust are composite indicators? Research Policy, 33(9), 1373–1384. https://doi.org/10.1016/j.respol.2004.09.007
\item Hall, J. L. (2008). Adding Meaning to Measurement. Economic Development Quarterly, 23(1), 3–12. https://doi.org/10.1177/0891242408326467
\item Hall, J. L. (2016). Developing Historical 50-State Indices of Innovation Capacity and Commercialization Capacity. Economic Development Quarterly, 21(2), 107–123. https://doi.org/10.1177/0891242406298128
\item Holbrook, A., \& Salazar, M. (2004). Regional Innovation Systems Within A Federation: Do national policies affect all regions equally? Innovation: Management, Policy \& Practice, 6(1), 50–64.
\item Isaksen, A. (2001). Building Regional Innovation Systems: Is Endogenous Industrial Development Possible in the Global Economy? Canadian Journal of Regional Science, 24(1), 101–120.
\item Pavitt, K., Robson, M., \& Townsend, J. (1987). The Size Distribution of Innovating Firms in the UK - 1945-1983. Journal of Industrial Economics, 35(3), 297–316.
\item Porter, M., \& Stern, S. (1999). The New Challenge to America’s Prosperity: Findings from the Innovation Index. Washington, D.C.: Council on Competitiveness.
\item Simmie, J. (2003). Innovation and urban regions as national and international nodes for the transfer and sharing of knowledge. Regional Studies, 37(6–7), 607–620. https://doi.org/10.1080/0034340032000108714
\item Soete, L. (2006). Knowledge, policy and innovation. In L. Earl \& F. Gault (Eds.), National Innovation, Indicators and Policy (pp. 198–218). Cheltenham: Edward Elgar.
\item Tijssen, R. (2003). Scoreboards of research excellence. Research Evaluation, 12(2), 91–103. https://doi.org/10.3152/147154403781776690
\end{enumerate}

\section{Annotated Bibliography \label{annotated_bib}}
\label{sec:org8457eae}


\bibliography{regional_innovation}
\bibliographystyle{plainnat}
\end{document}